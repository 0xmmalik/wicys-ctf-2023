\documentclass{article}
\usepackage{graphicx}
\usepackage{hyperref}
\usepackage{minted}

\title{WiCyS CTF 2023 Intelligent Curve Writeup}
\author{Manav Malik}
\date{}

\begin{document}

\maketitle

We begin, as always, by examining the challenge description:

\begin{quote}
    The other day, I was just looking at some words, and I found a five-letter word I thought was pretty cool! I did what any normal person does and encoded it using this elliptic curve.
    
    Consider the elliptic curve defined by $y^2 = x^3 + 16332764x + 8847792$ over a finite field of size $4952019383419$. Points $P$ and $Q$ on the curve are given below such that $kP = Q$ where $k$ is the message (encoded flag).
    
    $P = (637246119517, 588899099134)$
    
    $Q = (1241945849630, 4641791664291)$
    
    Your general solve path should be determining the value of $k$ and decoding the number into the word. The method used to encode the word into a number involves the ASCII values of each letter. The word consists of five letters ({\sf [a-zA-Z]\{5\}}). Remember to wrap the word in the {\sf WCS\{\}} format.
\end{quote}
The first step is, as the challenge description says, determining the value of $k$ that satisfies $kP=Q$. In general, trying to calculate $Q/P$ by hand is computationally very difficult and not a viable solution. Instead, we will need to exploit a major vulnerability of this elliptic curve.

One of the first things to check in any ECC challenge is the order of the elliptic curve, $\#E(F_p)$, where $F_p$ is a finite field of size $p$. The order of an elliptic curve is essentially equal to the number of reachable points on the curve. If $\#E(F_p)=p$ (the trace of Frobenius is 1), we can use an exploit known as \href{https://www.hpl.hp.com/techreports/97/HPL-97-128.pdf}{Smart's Attack}. We'll use SageMath for most of our calculations in this challenge. 

\begin{minted}{Sage}
    p = 4952019383419
    E = EllipticCurve(GF(p), [16332764, 8847792])
    # GF(n) defines a Galois Field of size n
    print(E.order() == p) # outputs True
\end{minted}
Now that we've determined that Smart's Attack can be applied, we'll use an implementation of it to determine the value of $k$.
\newpage
My personal implementation of Smart's Attack that I used to verify this challenge is messy and unreadable, so here I'm going to shamelessly steal Joachim Vandersmissen's implementation from his \href{https://github.com/jvdsn/crypto-attacks}{crypto-attacks repo on GitHub}.

\begin{minted}{Sage}
    def _lift(E, P, gf):
        x, y = map(ZZ, P.xy())
        for point_ in E.lift_x(x, all=True):
            _, y_ = map(gf, point_.xy())
            if y == y_:
                return point_


    def attack(G, P):
        E = G.curve()
        gf = E.base_ring()
        p = gf.order()
        assert E.trace_of_frobenius() == 1
    
        E = EllipticCurve(Qp(p),
            [int(a) + p * ZZ.random_element(1, p)
            for a in E.a_invariants()])
        G = p * _lift(E, G, gf)
        P = p * _lift(E, P, gf)
        Gx, Gy = G.xy()
        Px, Py = P.xy()
        return int(gf((Px / Py) / (Gx / Gy)))
\end{minted}
Running {\sf attack(P, Q)} yields $k\equiv76694417741\textbf{ mod }p$. Now, what remains is determining the word.

According to the challenge description, the word was encoded into a number with the ASCII values of each letter. These values will be two to three digits, so we should split our value for $k$ up into triplets (leading zeroes on numbers $<100$). (If you think all this is too guessy, this is quickly realized when we see that 76-69-44-177-41 becomes ``LE,±).'') Thus, we should expect our final number to be 15 digits long (or 14 if the first letter has an ASCII value $<100$). Because $k$ is a number modulo $p$, we can repeatedly add $p$ to our number and test it out.
\newpage
\begin{minted}{Python}
k = 76694417741
p = 4952019383419

alpha = "ABCDEFGHIJKLMNOPQRSTUVWXYZabcdefghijklmnopqrstuvwxyz"

while 15 - len(str(k)) > 1:
    k += p

valid = False
word = ""
while not valid:
    word = ""
    ks = str(k)
    chars = []
	
    if len(ks) == 14:
	chars = [ks[:2],ks[2:5],ks[5:8],ks[8:11],ks[11:]]
    else:
	chars = [ks[:3],ks[3:6],ks[6:9],ks[9:12],ks[12:]]
 
    valid = True
    for char in chars:
        c = chr(int(char))
	if c not in alpha:
		valid = False
	word += c
 
    k += p

print(word)
\end{minted}
In the end, the valid word is ``cuRVy,'' so our flag is thus: {\sf WCS\{cuRVy\}}.

\end{document}
